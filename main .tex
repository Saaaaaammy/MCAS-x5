\documentclass{article}
\usepackage{graphicx} % Required for inserting images
\usepackage{amssymb}
\usepackage{tikz}
\usepackage{amsmath}
\definecolor{CommentGreen}{rgb}{0.0,0.4,0.0}
\definecolor{Background}{rgb}{0.9,1.0,0.85}
\definecolor{lrow}{rgb}{0.914,0.918,0.922}
\definecolor{drow}{rgb}{0.725,0.745,0.769}

\usepackage{listings}
\usepackage{textcomp}
\lstloadlanguages{Python}%
\lstset{
    language=Python,
    upquote=true, frame=single,
    basicstyle=\small\ttfamily,
    backgroundcolor=\color{Background},
    keywordstyle=[1]\color{blue}\bfseries,
    keywordstyle=[2]\color{purple},
    keywordstyle=[3]\color{black}\bfseries,
    identifierstyle=,
    commentstyle=\usefont{T1}{pcr}{m}{sl}\color{CommentGreen}\small,
    stringstyle=\color{purple},
    showstringspaces=false, tabsize=5,
    morekeywords={properties,methods,classdef},
    morekeywords=[2]{handle},
    morecomment=[l][\color{blue}]{...},
    numbers=none, firstnumber=1,
    numberstyle=\tiny\color{blue},
    stepnumber=1, xleftmargin=10pt, xrightmargin=10pt
}
\usetikzlibrary{arrows.meta, positioning}
% Define the equation numbering format
\numberwithin{equation}{section}

\title{Group Coursework}
\author{Samuel Schofield, Terry McNally, Ethan Sloan}
\date{April 2024}

\begin{document}

\maketitle

\section{Introduction}
The aim of this project is to design a control system for the pitch of an aircraft. The control system diagram to get us started is shown in figure 1.\\\\
The input of the system is given by \(\delta\), the elevator deflection angle. Figure 1 also shows the angle of attack, \(\alpha\), the pitch angle \(\theta\) and the pitch rate \(r\).\\\\
Our Task is to do what boeing couldn't and design an MCAS system that is BIBO stable to keep the aircraft at a safe pitch and make sure that 737 max passengers only have to worry about doors falling off and not the aircraft nosediving.\\\\
In order to judge this, we have 3 criteria to assess the success of our control of the system.
\begin{itemize}
    \item \textbf{Stable} - Ensure the plane returns to zero degrees (meaning it has the correct pitch)
    \item \textbf{Zero Offset} - Ensure that the plane returns to the exact required pitch. In oscillations we'd like to see amplitudes of less than \(20^o\)
    \item \textbf{Limited Overshoot} - If the overshoot isn't adequately limited, not only would it be uncomfortable passengers, it could cause the system to be unstable in the opposite direction to the original 'kick'. This has obvious safety concerns and we don't want to cause a plane crash. Airbus flight control systems limit the pitch to \(30^o\) nose up and just \(15^o\) nose down so we thought that these would be good limits to work within.
\end{itemize}



\begin{figure}
    \centering
    \includegraphics[width=1\linewidth]{plen.png}
    \caption{Diagram of the assigned control system}
    \label{fig:Figure 1}
\end{figure}
\newpage
\section{The System}
The system dynamics are described by the following equations:
\begin{equation}
    \dot{\alpha}=-0.31\alpha + 56.7r + 0.231\delta
\end{equation}
\begin{equation}
    \dot{r}=-0.014\alpha - 0.426r +0.0203\delta
\end{equation}
\begin{equation}
    \dot{\theta} = 56.7r
\end{equation}
%
%
%
The sensor has the transfer function:
\begin{equation}
G_m(s) = exp(-0.005s)    
\end{equation}
Simulating a 5 ms Delay\\\\
%
\textbf{State Space Representation}:
\[
\begin{bmatrix} \dot{\alpha} \\ \dot{r} \\ \dot{\theta} \end{bmatrix} = \begin{bmatrix} -0.31 & 56.7 & 0.231\delta \\ -0.014 & -0.426 & 0.0203\delta \\ 0 & 56.7 & 0 \end{bmatrix} 
\]


\newpage
\section{Analysis - Deriving a Transfer Function}
We start by applying a Laplace transform to each of these equations:
\begin{equation}
\dot{\alpha} = -0.31\alpha + 56.7r + 0.231\delta \Rightarrow SA = -0.31A + 56.7R + 0.231\Delta
\end{equation}
\begin{equation}
\dot{r} = -0.014\alpha - 0.426r +0.0203\delta \Rightarrow SR = -0.014A - 0.426R +0.0203\Delta
\end{equation}
\begin{equation}
\dot{\theta} = 56.7r \Rightarrow S\Theta = 56.7R
\end{equation}

If we then make \(A\) the subject of equations [3.1] and [3.2] we get:

\begin{equation}
A(S + 0.31) = 56.7R + 0.231 \Delta \Rightarrow A = \frac{56.7R + 0.231 \Delta}{S + 0.31}
\end{equation}
\begin{equation}
0.014A = 0.0203\Delta - R(S + 0.426) \Rightarrow A = \frac{0.0203\Delta - R(S + 0.426)}{0.014}
\end{equation}

Since \(A=A\)
\begin{equation}
\therefore \frac{56.7R + 0.231 \Delta}{S + 0.31} = \frac{0.0203\Delta - R(S + 0.426)}{0.014}
\end{equation}

Collecting \(S\) terms:

\begin{equation}
0.7938R + 0.003234 \Delta = (S + 0.31)[0.0203\Delta - R(0.426 + S)]
\end{equation}

Simplifying the right-hand side (RHS):

\begin{equation}
RHS = (S + 0.31)[0.0203\Delta - R(0.426 + S)]
\end{equation}

\begin{equation}
= (S + 0.31)[0.0203\Delta - 0.426R - RS]
\end{equation}

Expanding the brackets:

\begin{equation}
= 0.0203\Delta S - 0.426RS - RS^2 + 0.006293\Delta - 0.13206R - 0.31RS
\end{equation}

Collecting like terms:

\begin{equation}
= 0.0203 \Delta S - 0.736RS - RS^2 + 0.006293\Delta - 0.13206R
\end{equation}

Bringing back the left side:

\begin{equation}
0.7938R + 0.736 RS + RS^2 = 0.0203\Delta S + 0.006293\Delta - 0.003234\Delta
\end{equation}

Collecting like terms:

\begin{equation}
0.92586R + 0.736 RS + RS^2 = 0.0203 \Delta S - 0.003056\Delta
\end{equation}

Factorizing out \(R\) and \(\Delta\):

\begin{equation}
R(0.92586 + 0.736S + S^2) = \Delta(0.0203S - 0.003056)
\end{equation}

Substituting in \(S\Theta = 56.7R\):

\begin{equation}
\frac{S\Theta}{56.7}(0.92586 + 0.736S + S^2) = \Delta(0.0203S - 0.003056)
\end{equation}

\begin{equation}
\frac{56.7(0.0203S - 0.003056)}{S\Theta(0.92586 + 0.7365S + S^2)} = \frac{\Theta}{\Delta}
\end{equation}

Therefore, final transfer function:

\begin{equation}
\frac{\Theta}{\Delta} = \frac{1.15101S - 0.1732752}{0.92586S + 0.736S^2 + S^3}
\end{equation}

Find poles to test for BIBO stability:

\begin{equation}
S^3 + 0.736S^2 + 0.92586S = 0
\end{equation}

\begin{equation}
S(S^2 + 0.736S + 0.92586) = 0
\end{equation}

Therefore, \(S = 0\).

Using the quadratic formula for remaining roots:

\begin{equation}
S = \frac{-0.736 \pm \sqrt{0.736^2 - 4(1)(0.92586)}}{2(1)}
\end{equation}

\begin{equation}
S = \frac{-0.736 \pm \sqrt{-3.161744}}{2}
\end{equation}

\begin{equation}
S = -0.368 \pm 0.5j \sqrt{3.161744}
\end{equation}

Therefore, poles of \(S\):

\begin{equation}
S = 0, S= -0.368 + 0.5j \sqrt{3.161744}, S = -0.368 - 0.5j \sqrt{3.161744}
\end{equation}
Therefore, since all poles are negative on the 'real axis' (left of the origin), this system must be stable.\\\\
\textbf{For the closed loop transfer function:}
\begin{equation}
    \frac{K_c}{\frac{0.92586S + 0.736S^2 + S^3}{1.15101S - 0.1732752} + K_c} \times \frac{1.15101S - 0.1732752}{1.15101S - 0.1732752}
    = \frac{K_c (1.15101S - 0.1732752)}{0.92586S + 0.736S^2 + S^3 + K_c(1.1501S - 0.1732752}
\end{equation}
\begin{equation}
    = \frac{K_c (1.15101S - 0.1732752)}{0.92586S + 0.736S^2 + S^3 + 1.1510K_c S - 0.1732752K_c}
\end{equation}
\begin{equation}
     = \frac{K_c (1.15101S - 0.1732752)}{S^3 + 0.736S^2 + S(0.92586+1.15101K_c) - 0.1732752K_c}
\end{equation}

\newpage
\section{Control - Forming a PID Controller}
\begin{figure}
    \centering
    \includegraphics[width=1\linewidth]{Block Diagram.jpg}
    \caption{PD Controller block diagram}
    \label{fig:enter-label}
\end{figure}
Here is the code we used to form our PID controller using our transfer function and adding the sensor delay:
\begin{lstlisting} [language=python]
#This is the delay for our sytem to react
delay=np.exp(5e-3)

Gpid=kp+ki/s+kd*s*(-delay*s)

G_load = ctrl.feedback(G, Gpid, sign=-1)

t_imp, theta_imp=ctrl.impulse_response(G_load)

#below is the code to plot a graph
plt.plot(t_imp, np.rad2deg(theta_imp))

plt.grid(True)

plt.show

\end{lstlisting}
%___
To find the closest to ideal PID parameters, we used the Ziegler - Nichols method.
This involves finding the largest value of P, whilst I and D are zero, where the graph oscillates whilst remaining stable (our ultimate gain). The period of one oscillation is then known as the ultimate period.\\
We found our ultimate gain value to be 50 and ultimate period value to be 1.215.\\\\
%
%
This formed a set of PD values:\\
\(kP = 0.8(50) = 40\)\\
\(kD = 1.215/8 = 0.151875\)\\
As shown in figure 3, the graph formed was perfect up till 20 seconds where the angle dropped rapidly. This is unfortunately therefore not stable and cannot be the solution we use.

\newpage
\[\]\\
We then moved to a more trial and error approach where we tested very high, very low and values closer to zero in order to find the ideals.
The most ideal values we found were as follows:\\
\[kP = 5\times10^{12}\]
\[kI = 0.1\]
\[kD = 0.1\]
This produced the graph in Figure 4.\\\\
This graph ticks all our Criteria:
\begin{enumerate}
\item [\checkmark] \textbf{Stable} - The graph remains stable and does not have massive increases or decreases in angle. 
\item [\checkmark] \textbf{Zero Offset} - The pitch angle returns to zero after just 12 seconds
\item [\checkmark] \textbf{Limited Overshoot} - The minimum and maximum angle stay well within our limits and don't even reach \(-2^o\) and \(+3^o\) respectively  
\end{enumerate}
%
However we are unsure whether a proportional value of five trillion is realistic as it may be impossible to implement. However in are testing we were unable to find another value that fulfilled all of our criteria as many numbers tended to have an exponential decline in angle after (at best) a few minutes.

\begin{figure}
    \centering
    \includegraphics[width=0.5\linewidth]{FIG 3.png}
    \caption{Graph obtained using Ziegler-Nichols method.}
    \label{fig:enter-label}
\end{figure}
\begin{figure}
    \centering
    \includegraphics[width=0.5\linewidth]{FIG 4.png}
    \caption{Final Graph}
    \label{fig:enter-label}
\end{figure}
\newpage
\section{Team Collaboration}
Our team collaborated well, mainly through face to face meetings (the minutes from which will be on the GitHub). We used GitHub to share the code and overleaf's collaboration tools to see an up to date version of the report which we also uploaded to GitHub at regular intervals. \\\\
\textbf{What Went Well?}\\
Work was split well amongst the group with members taking a lead where their strengths were. Despite this everyone did a bit of everything in order to make sure that the workload was never to high for one person. The face to face nature of our meetings helped this and allowed us to set a goal for the end of each day.
General roles:
\begin{itemize}
    \item Ethan: As the strongest in the group at derivation, he took lead on transfer function derivation and PID controller coding and testing.
    \item Sammy: Due to LaTex experience gained from doing his lab report, he took a lead here. He also set up and managed the GitHub tasks and helped with maths when needed.
    \item Terry: Worked out PID values for testing, checked for BIBO stability using Routh's tabulation, also did derivations and maths throughout the project.
\end{itemize}
\textbf{Challenges}\\
The group collaborated well, we set a large amount of time for the project and therefore were able to meet and work together effectively. The two largest issues were google colab's tendency to lag when trying to run code and the fact that the group project rooms close at 5pm!

\newpage
\section{Conclusion}
In conclusion, stabilising this system within our criteria was harder than we thought. We got quite far with the Ziegler - Nichols method. However unfortunately the model's tendency to nosedive also affected it here.\\ 
Ultimately we have built a controller that does fulfill all of our criteria as shown above even if implementation of it may not be realistic.
\end{document}
